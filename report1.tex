\documentclass[pdftex,12pt,a4paper]{report} 

% Document settings

\usepackage{fullpage}
\usepackage{cite}
\usepackage{datetime} 
\usepackage{geometry}
\usepackage[pdftex]{graphicx}
\usepackage{verbatim}
\usepackage{todonotes}

\geometry{verbose,lmargin=3cm,rmargin=3cm}

\newcommand{\HRule}{\rule{\linewidth}{0.5mm}}

\begin{document}

\title{Computer Music Improvisation - Grammatical Approach For Blues}
\author{Cliff Sun (chs09)}
\date{November 2012}
\maketitle

\begin{abstract}

Write the abstract last.



Improvisation is a very interesting topic in the world of music. There are many factors that influence how a musician improvises with an instrument. In addition blues has simple rules and allows the player to take any sort of musical direction they want. Here we will be looking at a grammatical and symbolic approach to generating blues music on top of a  simple twelve bar blues accompaniment. 

\end{abstract}

\setcounter{tocdepth}{2} % Set the depth of toc indexing

\tableofcontents

\pagebreak

\renewcommand*\thesection{\arabic{section}}

\section{Acknowledgements}

acknowledge

\pagebreak

\chapter{Introduction}

\section{Improvisation}

Interesting questions we would like to answer with our project are:
\begin{itemize}

\item blues

\item music

\item mozart

\end{itemize}

\section{section}

something here

\paragraph{paragraph}
para

\subsection{subsection}

subsection within section

\pagebreak



\end{document}